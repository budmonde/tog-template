%
% Colors
%
\definecolor{INCOMPLETECOLOR}{RGB}{178,34,34}
\definecolor{UNDERREVISIONCOLOR}{RGB}{210,121,121}
\definecolor{FEEDBACKNEEDEDCOLOR}{RGB}{230,170,50}
\definecolor{FEEDBACKGIVENCOLOR}{RGB}{121,210,121}
\definecolor{COMPLETECOLOR}{RGB}{121,124,210}
\definecolor{LOCKEDCOLOR}{RGB}{153,102,255}

\definecolor{TODOCOLOR}{RGB}{255,0,0}
\definecolor{MONDECOLOR}{RGB}{0,0,255}
\definecolor{QICOLOR}{RGB}{0,255,0}
\definecolor{ANJULCOLOR}{RGB}{127,127,0}
\definecolor{RACHELCOLOR}{RGB}{127,0,127}
\definecolor{GUESTCOLOR}{RGB}{0,127,127}

\definecolor{WHITE}{RGB}{255,255,255}

%
% Global macros
%
\newcommand{\nothing}[1]{}
\newcommand{\isolated}[1]{\hfill\break#1\hfill}
\newcommand{\Caption}[2]{\caption[#1]{{\em #1} #2}}
\newcommand{\Quote}[1]{{``#1''}}

\newcommand{\Ie}{{I.e.,}\xspace}
\newcommand{\ie}{{i.e.,}\xspace}
\newcommand{\Eg}{{E.g.,}\xspace}
\newcommand{\eg}{{e.g.,}\xspace}
\newcommand{\etal}{{et~al\xperiod}\xspace}
\newcommand{\aka}{{a.k.a.,}\xspace}
\newcommand{\etc}{{etc\xperiod}\xspace}
\newcommand{\cf}{{cf\xperiod}\xspace}

%
% Adding Comments
%
\ifthenelse{\equal{\status}{0}} {                     % INTERNAL Version

    \newcommand{\todo}[1]{
        \addcontentsline{toc}{subsection}{            %
            \protect\numberline{}                     % Align text appropriately
            \textcolor{TODOCOLOR}{[TODO] #1}}         % Add todo to the table of contents
        \isolated{
            \textcolor{TODOCOLOR}{[TODO] \emph{#1}}}} % Typeset the todo note in the text
    \newcommand{\warning}[1]{\todo{#1}}
    \newcommand{\note}[1]{{\it\color{blue} #1}}

    \newcommand{\todolist}{\tableofcontents}

    % Usage: \monde[(OPTIONAL) Text to underline]{Comment.}
    \newcommandx{\monde}[2][1=]
        {\setulcolor{MONDECOLOR}{\ul{#1}}
         \isolated{\textcolor{MONDECOLOR}{\textbf{Monde:} #2}}}
    \newcommandx{\qi}[2][1=]
        {\setulcolor{QICOLOR}{\ul{#1}}
         \isolated{\textcolor{QICOLOR}{\textbf{Qi:} #2}}}
    \newcommandx{\anjul}[2][1=]
        {\setulcolor{ANJULCOLOR}{\ul{#1}}
         \isolated{\textcolor{ANJULCOLOR}{\textbf{Anjul:} #2}}}
    \newcommandx{\rachel}[2][1=]
        {\setulcolor{RACHELCOLOR}{\ul{#1}}
         \isolated{\textcolor{RACHELCOLOR}{\textbf{Rachel:} #2}}}

    % Usage: \guest[(OPTIONAL) Text to underline]{Name}{Comment.}
    \newcommandx{\guest}[3][1=]
        {\setulcolor{LOCKEDCOLOR}{\ul{#1}} \textcolor{LOCKEDCOLOR}
        {[\textbf{#2:} #3]}}
}{                                                    % NON-INTERNAL Versions
    \newcommand{\todo}[1]{}
    \newcommand{\warning}[1]{}
    \newcommand{\note}[1]{}
    \newcommand{\todolist}{}
    \newcommandx{\monde}[2][1=]{#1}
    \newcommandx{\qi}[2][1=]{#1}
    \newcommandx{\anjul}[2][1=]{#1}
    \newcommandx{\rachel}[2][1=]{#1}
    \newcommandx{\guest}[3][1=]{#1}
}

%
% Revision Annotation
% Usage: \revise{Text to delete.}{Text to insert.}
%

\ifthenelse{\equal{\status}{0}} {          % INTERNAL Version
    % Highlight revised text.
    % Strike-through deleted text
    \newcommand{\revise}[2]{\textcolor{INCOMPLETECOLOR}{\st{#1}}\textcolor{blue}{#2}}
}{}
\ifthenelse{\equal{\status}{1}} {          % SUBMISSION Version
    \newcommand{\revise}[2]{#2}
}{}
\ifthenelse{\equal{\status}{2}} {          % REVISION Version
    % Highlight revised text.
    \newcommand{\revise}[2]{\textcolor{blue}{#2}}
}{}
\ifthenelse{\equal{\status}{2}} {          % CAMERA-READY Version
    \newcommand{\revise}[2]{#2}
}{}
\ifthenelse{\equal{\status}{2}} {          % CAMERA-READY AUTHOR Version
    \newcommand{\revise}[2]{#2}
}{}

%
% Status Badges
%
\ifthenelse{\equal{\status}{0}} {
    \newcommand{\badge}[2]{\colorbox{#1}{\small\textcolor{WHITE}{\texttt{#2}}}}
    \newcommand{\headerBadge}[2]{\hspace*{\fill}\badge{#1}{#2}}
    \newcommand{\locked}{\headerBadge{LOCKEDCOLOR}{locked}}
    \newcommand{\complete}{\headerBadge{COMPLETECOLOR}{complete}}
    \newcommand{\feedbackGiven}{\headerBadge{FEEDBACKGIVENCOLOR}{feedback given}}
    \newcommand{\feedbackNeeded}{\headerBadge{FEEDBACKNEEDEDCOLOR}{feedback needed}}
    \newcommand{\underRevision}{\headerBadge{UNDERREVISIONCOLOR}{under revision}}
    \newcommand{\incomplete}{\headerBadge{INCOMPLETECOLOR}{incomplete}}
}{
    \newcommand{\badge}[2]{}{}
    \newcommand{\headerBadge}[2]{}{}
    \newcommand{\locked}{}
    \newcommand{\complete}{}
    \newcommand{\feedbackGiven}{}
    \newcommand{\feedbackNeeded}{}
    \newcommand{\underRevision}{}
    \newcommand{\incomplete}{}
}
