\section{Introduction \incomplete}
\label{sec:introduction}

\note{
What problem we are trying to solve.
Why it is important, and why people should care.
}%note

Writing research documents, such as conference/journal/white papers, patent applications, grant proposals, and course reports, is a core activity for many poor souls including professors, researchers, engineers, and students.
Since we must do it one way or another, sooner or later, we might as well do it as happily and effectively as possible.
This document is meant to document both my experience of learning this process and also share my findings on effective writing advice and general project management.
Note that large portions of this document has been inherited from my advisers and mentors and developed further (see Acknowledgements).

\note{
What prior works have done, and why they are not adequate.
(Note: this is just high level big ideas. Details should go to a previous work section.)
}%note

Some people, including very successful ones, write papers only at the end of a project, like days if not hours prior to a submission deadline.
This almost always leads to total chaos and breakdown, unless you have other means to keep track and organize all the relevant pieces of information.

\note{
What our method has to offer, sales pitch for concrete benefits, not technical details.
Imagine we are doing a TV advertisement here.
}%note

This document provides a template for writing research papers, and describes how to manage project progress via iterating drafts.
Doing this correctly can help you achieve productive research and a happy life, e.g. working anywhere anytime without synchronous meetings with your collaborators.

\note{
Our main idea, giving people a take home message and (if possible) see how clever we are.
}%note

I learned from my PhD adviser to start writing from the moment I have the faintest idea, and gradually update the drafts to reflect progress.
Coming from a software engineering background, I was aware of how revision control is essential for sharing and editing source files, especially among large groups of collaborators.
The iterative nature of writing and coding matches well with the capability of revision control.
Thus I prefer Latex + git/svn (e.g. github, bitbucket, and the svn server under my hosting service), even though I can still manage with MS Word + cloud drive (e.g. Google drive, Box, and Dropbox). 
A good computer scientist writes papers like programs, and manages Latex files like source codes.
These repos are external memories and communication mediums for the collective brains of my teams.

\note{
Our algorithms and methods to show technical contributions and that our solutions are not trivial.
}%note

\begin{figure}
  \centering
  \includegraphics[width=\linewidth]{example-image-a}
  \Caption{
      Figure title.
  }{
      More detailed description of figure.
  }
  \Description{Description of figure.}
  \label{fig:example}
\end{figure}


Translating thoughts into words, diagrams, equations, pseudo-codes, or codes is a good way to clarify and consolidate.
If you are writing a graphics/HCI paper, design the figures so that they alone can provide a high level picture of the main points.
As exemplified in \Cref{fig:example}, a figure can contain images, drawings, or combinations. 

\note{
Results, applications, and extra benefits.
}%note

You can use the associated files as a starting template for your research papers.
Start with the \texttt{Makefile}, and notice the two build targets: final for official submission and public disclosure, and draft for internal sharing among your collaborators (including yourself).
